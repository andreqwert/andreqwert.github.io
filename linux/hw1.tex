\documentclass[a4paper, 14pt]{article}
\usepackage[T2A]{fontenc}
\usepackage[utf8]{inputenc}
\usepackage{mathtext}
\usepackage[english, russian]{babel}
\usepackage{euscript}
\usepackage{mathrsfs}
\usepackage{lscape}
\usepackage{cmap}
\usepackage{listings}
\usepackage{indentfirst} % чтобы первый абзац был с отступом.
\usepackage[14pt]{extsizes}
\usepackage{amsmath}
\usepackage{subfigure}
\usepackage{xcolor}
\usepackage{hyperref}
\usepackage{color} %% это для отображения цвета в коде
\usepackage{listings} %% собственно, это и есть пакет listings
\usepackage{caption}
\usepackage{multirow} 
\usepackage[russian]{babel}


\DeclareCaptionFont{white}{\color{white}} %% это сделает текст заголовка белым
%% код ниже нарисует серую рамочку вокруг заголовка кода.
\DeclareCaptionFormat{listing}{\colorbox{black}{\parbox{\textwidth}{#1#2#3}}}
\captionsetup[lstlisting]
{format=listing,labelfont=white,textfont=white}
\definecolor{linkcolor}{HTML}{799B03} % цвет ссылок
\definecolor{urlcolor}{HTML}{799B03} % цвет гиперссылок
\usepackage[pdftex]{graphicx}
\graphicspath{{pictures/}}
\DeclareGraphicsExtensions{.png,.jpg}




\author{}
\title{}
\date{}  % Чтобы убрать дату
\usepackage{geometry} % Меняем поля страницы
\geometry{left=2cm}% левое поле
\geometry{right=0.5cm}% правое поле
\geometry{top=1cm}% верхнее поле
\geometry{bottom=2cm}% нижнее поле

\begin{document}



	\pagestyle{plain} % нумерация вкл.
	
	\newpage
	
	\section*{Домашнее задание №1}
	
	\subsection*{Постановка задачи}
	
	Установить VirtualBox (последний), чтобы у всех был одинаковый софт с одинаковыми настройками.
	
	Установить в виртуалку \textbf{(2Core, 1G RAM, 8G vhdd) Ubuntu 16.04.2 (разрядность не важна, но желательно amd64)}. Количество ядер и размер оперативки можно поправить позже.

При установке добавить роль openssh-server, в остальном - minimal, т.е. base system + openssh-server.

После установки сделать Snapshot VM средствами VBox.
	
	\subsection*{Процесс решения}
	
	При помощи SSH можно удалённо из командной строки MacOS управлять Ubuntu (и другими Linux). Управлять, вбивая команды в Терминале, можно до тех пор, пока виртуальная машина не будет выключена.
	
	\vspace{1.5cm}
	
	Логика решения следующая:
	
	1. Устанавливаем VirtualBox для Mac.
	
	2. Скачиваем Ubuntu 16.04.2, устанавливаем ее на виртуальную машину с настройками, данными в постановке задачи (выделены жирным шрифтом).
	
	3. VirtualBox --> <<Настроить>> (жёлтая кнопка) --> Сеть --> выбрать <<Сетевой мост>> и сеть en1:Wi-fi (Airport). Нажать <<ОК>>.
	
	4. Запустить Ubuntu. Войдя в терминал, набрать \textit{ifconfig | grep addr}. В третьей строчке будет что-то вроде \textit{inet addr:192.168.1.39 <что-то там>}. Это -- адрес виртуальной машины. К нему будем подключаться.
	
	5. Вводим в том же терминале Ubuntu команду: \textit{sudo apt-get install openssh-server}.
	
	6. Теперь заходим с терминала MacOS и подключаемся посредством команды \textit{ssh user@192.168.1.39} (вместо слова user пишется имя пользователя).
	
	7. В VirtualBox нажать Machine --> Take Snapshot. 
	
	8. Profit!
	
	\vspace{1.5cm}
	
	Подобная операция под Windows делается при помощи виндовой утилиты PUTTY. Так вот, зачем она нужна!
	
	Видео по настройке для Windows: \href{https://www.youtube.com/watch?v=5BsShkcweIs}{https://www.youtube.com/watch?v=5BsShkcweIs}


\end{document} 