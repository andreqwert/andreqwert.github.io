\documentclass[a4paper, 14pt]{article}
\usepackage[T2A]{fontenc}
\usepackage[utf8]{inputenc}
\usepackage{mathtext}
\usepackage[english, russian]{babel}
\usepackage{euscript}
\usepackage{mathrsfs}
\usepackage{lscape}
\usepackage{cmap}
\usepackage{listings}
\usepackage{indentfirst} % чтобы первый абзац был с отступом.
\usepackage[14pt]{extsizes}
\usepackage{amsmath}
\usepackage{subfigure}
\usepackage{xcolor}
\usepackage{hyperref}
\usepackage{color} %% это для отображения цвета в коде
\usepackage{listings} %% собственно, это и есть пакет listings
\usepackage{caption}
\usepackage{multirow} 
\usepackage[russian]{babel}


\DeclareCaptionFont{white}{\color{white}} %% это сделает текст заголовка белым
%% код ниже нарисует серую рамочку вокруг заголовка кода.
\DeclareCaptionFormat{listing}{\colorbox{black}{\parbox{\textwidth}{#1#2#3}}}
\captionsetup[lstlisting]
{format=listing,labelfont=white,textfont=white}
\definecolor{linkcolor}{HTML}{799B03} % цвет ссылок
\definecolor{urlcolor}{HTML}{799B03} % цвет гиперссылок
\usepackage[pdftex]{graphicx}
\graphicspath{{pictures/}}
\DeclareGraphicsExtensions{.png,.jpg}




\author{}
\title{}
\date{}  % Чтобы убрать дату
\usepackage{geometry} % Меняем поля страницы
\geometry{left=2cm}% левое поле
\geometry{right=0.5cm}% правое поле
\geometry{top=1cm}% верхнее поле
\geometry{bottom=2cm}% нижнее поле

\begin{document}



	\pagestyle{plain} % нумерация вкл.
	
	\newpage
	
	\section*{Домашнее задание №2}
	
	\subsection*{Постановка задачи}
	
	В домашнем каталоге создать каталог \textbf{f.lastname} (\textit{f} -- первая буква имени). В этом каталоге:
	
	1. Вывести список процессов в \textit{ps.out};
	
	2. Сохранить информацию о доступной памяти и дисковом пространстве в файлы \textit{mem.out} и \textit{disk.out};
	
	3. Создать структуру каталогов:
	
	\hspace{0.2cm}1. Dir1
	
	\hspace{1.2cm} 1. Dir3, Dir4, Dir5
	
	\hspace{0.2cm}2. Dir2
	
	\hspace{1.2cm} 1. Dir6, Dir7, Dir8	
	
	4. Сохранить последовательность действий при помощи \textit{history > history.out};
	
	5. Получившийся \textit{history.out} переименовать в \textit{f.lastname.out}.
	
	\subsection*{Процесс решения}
	
	Запускаем виртуальную машину с линуксом. Всё, больше линукс не нужен.
	
	\vspace{0.1cm}
	
	На MacOS открываем iTerm2. \textbf{Подключаемся к виртальной машине}. Подробнее про ssh в ДЗ № 1:
	
	\textit{ssh user@192.168.1.39}
	
	\vspace{0.2cm}
	
	 Создаём папку и переходим в неё:
	
	\textit{mkdir f.lastname}
	
	\textit{cd f.lastname}
	
	\vspace{0.2cm}
	
	\textbf{Сохрaняем список процессов в файл} \textit{ps.out}:
	
	\textit{ps > ps.out}
	
	\vspace{0.2cm}
	
	\textbf{Сохраняем информацию о дисковом пространстве в файл} \textit{disk.out}:
	
	\textit{df > disk.out}
	
	\vspace{0.2cm}
	
	А \textbf{информацию о доступной памяти -- в файл} \textit{mem.out}:
	
	\textit{free > mem.out}
	
	\vspace{0.2cm}
	
	Создаём структуру каталогов:
	
	\textit{mkdir Dir1}
	
	\textit{cd Dir1}
	
	\textit{mkdir Dir3}
	
	\textit{mkdir Dir4}
	
	\textit{mkdir Dir5}
	
	\vspace{0.2cm}
	
	\textbf{Возврат на каталог ниже}:
	
	\textit{cd ..}
	
	\vspace{0.2cm}
	
	И дальше создаём структуру каталога:
	
	\textit{mkdir Dir2}
	
	\textit{cd Dir2}
	
	\textit{mkdir Dir6}
	
	\textit{mkdir Dir7}
	
	\textit{mkdir Dir8}

	\vspace{0.2cm}
	
	Возврат на каталог ниже:
	
	\textit{cd ..}
	
	\vspace{0.2cm}
	
	\textbf{Сохраняем историю действий} в терминале:
	
	\textit{history > history.out}
	
	\vspace{0.2cm}
	
	\textbf{Переименовываем} получившийся файл:
	
	\textit{mv history.out f.lastname.out}
	
	\vspace{3cm}
	
	Полезным может также оказаться \textbf{очищение истории команд} терминала:
	
	\textit{history -cw}
	
	\vspace{0.2cm}

	\textbf{Запуск файла}:
	
	\textit{cat f.lastname.out}
	
	
	
	
	
	
	

\end{document} 